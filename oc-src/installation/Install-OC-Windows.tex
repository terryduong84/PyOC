\documentclass[12pt]{article}
\usepackage[latin1]{inputenc}
\usepackage{graphicx,subfigure}
\topmargin -1mm
\oddsidemargin -1mm
\evensidemargin -1mm
\textwidth 165mm
\textheight 220mm
\parskip 2mm
\parindent 3mm
%\pagestyle{empty}

\begin{document}
\begin{center}
{\Large \bf Installation of OpenCalphad on Windows}

Bo Sundman, \today

\end{center}

There is no automatic installation routine for OC, you must compile
the software yourself after downloading.  You may also have to install
Fortran and C compilers and the GNUPLOT software if you do not already
have them.  The OC development team cannot offer you any help for
that, please ask some local experts if you need help.

On Windows you can use a native Fortran compiler like Microsoft
Developer Studio or Intel but there is no installation guide for that.
Maybe you can use some of the information below.  Usually you have to
pay for these so you may be able to get some help from the vendor if
you have problems.

You can also use the free MSYS2/MinGW or Cygwin system which emulates
(in slightly different ways) a linux OS on Windows.  This guide is for
MSYS2/MinGW.  There is a French guide for Cygwin which is more
elaborate (and it does not require that you understand much French).

The descrcipton below applies when installing OC on Windows using
MSYS2/MinGW, the other guides available are (probably not updated):
\begin{itemize}
\item Install-OC-parallel-Windows-French
\item Install-OC-Linux-or-macOS
\end{itemize}

Step by step installation:

\begin{enumerate}
\item Download OC from http://opencalphad.org or the sundmanbo
  repository at http://github.com to some directory on your computer.

\item The OC software is written in the new Fortran standard and
  requires a compiler like GNU Fortran 4.8 or similar.  It also uses
  some library routines written in C.

\item If you have not already installed MSYS2/MinGW and the Fortran
  compiler you must do that from https://SourgeForge.net or some
  similar site.  If you have MSYS2/MinGW but not the Fortran compiler
  you must add that.  This software is free.

\item Open a terminal window.  If you do not know what is a terminal
  window on Windows you should ask a local expert.  Keep him or her
  with you until you finished the installation.

\item In the terminal window you may have to use ``cd'' (change
  directory) until you reach the direcory where you dowloaded OC.

\item Unzipp OC, this will create a directory opencalphad-master or
  opencalphad-versionX.

\item cd to this directory in your terminal window.

\item It will also be convenient to go to this directory in a file
  explorer window.  You will find several files and directories.
  There is a readme.pdf file which tells you to look in the
  installation directory for this file.

\item In this directory rename the file ``linkmake'' to linkmake.cmd
  so it can be executed.

\item If your computer have several kernels you can use OC with
  parallelization using Open MP.  In that case you should use the
  linkfile ``linkpara'' below (after renaming it to linkpara.cmd).

\item Then exectute the file you just renamed by typing its name in
  the terminal window.

\item {\bf If you have errors running the linkmake or linkpara command
  files please ask a local expert for help.}

\item From version 5.018 OC has a popup window for opening files to
  make it easy to search for files.  This is in the
  ``tinyfiledialogs'' which is compiled together with OC.  If you have
  problems with this or do not want this feature try compiling using
  ``linkmake-notinyfd'' or ``linkpara-notinyfd'' instead.

\item For the graphics you must download and install the free GNUPLOT
  software, you can find that on SoureForge.

  Make sure your PATH includes the directory with the GNUPLOT program.
  You can check this by simply typing GNUPLOT in the terminal window.
  If you get the message ``gnuplot is not recognized ...'' ask your
  local expert to fix the path or installation of gnuplot.

\item The steps below are not necessary but may enhance the use of OC
  by creating a home directory for OC
  \begin{itemize} 
    \item Create a directory called OCHOME at you home directory,
      usually\\
      ``C:${\backslash}$Users${\backslash}$yourname''.

    \item Create an environment variable for your account called
      OCHOME with the path to your OCHOME directory as value.  If you
      do not know how to create an environment variable please ask a
      local expert.
      
% On my Swedish Windows system I create a environment variable by
% clicking on the Windows icon and then type ``milj�'' and select
% ``Redigera milj�variabler f�r kontot''

      Normally you have to restart your computer to make the
      environment variable available.

    \item Copy the file ochelp.hlp to this directory
    
    \item Later you may also add a macro file on this directory called
      ``start.OCM'' that will be run everytime you start OC.  You may
      also create a direcory called ``databases'' with databases you
      use.  Such databases will be searched if you prefix the database
      name with ``ocdata/'' in the command ``read tdb''

    \item If you want to start the OC program from any directory copy
      also the executable to OCHOME and add the path to OCHOME to your
      \%PATH\%
  \end{itemize}

\item Look in ``after-installation'' for help to use OC.

\end{enumerate}

You are welcome to help providing a better installation guide also!

\bigskip

{\large \bf Have fun and help make OC useful!}

\end{document}

